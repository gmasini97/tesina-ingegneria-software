\documentclass{article}
\usepackage[italian]{babel}
\usepackage{titling}
\usepackage[style=ieee]{biblatex}
\usepackage{hyperref}
\usepackage{textcomp}
\usepackage{tabularx}
\hypersetup{
    colorlinks,
    citecolor=black,
    filecolor=black,
    linkcolor=black,
    urlcolor=black
}

\addbibresource{108456-masini.bib}

\title{Progetto senza titolo}
\author{Gabriele Masini \#108456}
\date{2019-12-10}

\begin{document}
    
\maketitle
\pagenumbering{gobble}
\newpage
\tableofcontents
\newpage
\pagenumbering{arabic}
\section{Preambolo}
Grazie alla diffusione dell'\textit{Internet of Things} e con l'introduzione sul mercato di dispositivi come \textit{Amazon}\textregistered \textit{Echo}\texttrademark si sono resi disponibili al consumatore una moltitudine di prodotti \textit{smart} tra cui telecamere ip, videocitofoni e sensori di vario genere atti a "digitalizzare" la propria esperienza casalinga.
Che si critichi o no la cosa, questa diffusione di dispositivi rende disponibile all'uomo tecnologie multimediali a basso prezzo.
È esattamente in questo contesto che si inserisce il progetto \textit{\thetitle}, il quale si pone come obbiettivo la realizzazione di un sistema di gestione e registrazione per sistemi di sicurezza e/o videosorveglianza dei propri dispositivi \textit{smart} installati nella propria abitazione.
\section{Introduzione}
\subsection{Scopo}
Il presente documento si riferisce alla definizione e descrizione delle informazioni necessarie per lo sviluppo del progetto \textit{\thetitle} e interessa tutte le entità che siano coinvolte nella progettazione, sviluppo e utilizzo dello stesso. Il documento viene redatto secondo le direttive espresse nel \textit{IEEE Recommended Paractice for Software Requirements Specifications} \cite{ieeestd830}, ovvero secondo lo standard \textit{IEEE Std 830-1998}.

\subsection{Campo di applicazione}
Il progetto nominato \textit{\thetitle} si inserice nel contesto odierno di forte sviluppo di tecnologie relative all'\textit{Internet of Things} e si pone come obbiettivo la gestione dei dispositivi \textit{smart} multimediali per scopi di monitoraggio e videosorveglianza della propria abitazione, nonché fornire dei metodi comuni di registrazione dei sopra menzionati dispositivi. \textit{\thetitle} non si interfaccerà utilizzando protocolli proprietari dei singoli produttori, ma utilizzerà protocolli comuni e di pubblico dominio di utilizzo, incluso, ma non limitato a, \textit{http, rtsp, udp}.
\subsection{Definizioni, acronimi, abbreviazioni}
Nel presente documento vengono spesso riportate abbreviazioni e acronimi comuni nel contesto di sviluppo del progetto. Di seguito vengono riportate le interpretazioni delle suddette abbreviazioni:
\begin{center}
    \begin{tabularx}{\linewidth}{c X}
            IOT & Internet of Things, l'aggregazione in internet di dispositivi di poca capacità di calcolo destinati come controllori per determinati elettrodomestici o utilizzati come sensori (telecamere, termometri, microfoni etc). \\
            Ipcam & Telecamera, spesso utilizzata per videosorveglianza, che rende disponibile verso l'esterno una interfaccia ip per la gestione della stessa e la visione dello stream video tramite un protocollo comune (rtsp, udp etc). \\
            Stream & Flusso di dati di vario tipo (audio, video, audiovideo etc).
    \end{tabularx}
\end{center}
\subsection{Riferimenti}
\printbibliography[heading=none]
\subsection{Vista d'insieme del documento}
Il presente documento contiene le specifiche e le relative descrizioni dei requisiti del progetto \textit{\thetitle}. Il progetto viene inquadrato secondo quanto citato in questa sezione e si prosegue nella sezione successiva a descriverne la prospettiva e le funzionalità, nonché la caratterizzazione dei potenziali utenti, dei vincoli e delle dipendenze. Nella sezione \textit{\hyperref[sec:specifica]{Specifica dei requisiti}}, verranno spiegate le interfacce con l'esterno e i requisiti funzionali e non.
Alla fine del documento viene proposto in \textit{\hyperref[sec:appendice]{Appendice}} un possibile utilizzo dei celeberrimi \textit{design pattern} \cite{designpatterns} noti per la programmazione ad oggetti del progetto.
\section{Descrizione generale}
\subsection{Prospettiva del prodotto}
Il progetto \textit{\thetitle} 
\subsection{Funzionalità del prodotto}
\subsection{Caratteristiche degli utenti}
\subsection{Vincoli e limiti}
\subsection{Presupposti e dipendenze}
\section{Specifica dei requisiti}
\label{sec:specifica}
\subsection{Interfacce con l'esterno}
\subsection{Requisiti funzionali}
\subsection{Requisiti non funzionali}
\section{Appendice}
\label{sec:appendice}


\end{document}